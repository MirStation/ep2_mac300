\documentclass[a4paper,11pt]{article}
\usepackage{times}
\usepackage{listings}
\usepackage{amsmath}
\usepackage[top=10mm, bottom=15mm, left=20mm, right=20mm]{geometry}
\usepackage{multirow}
\usepackage{hhline}

%% Escrevendo em português
\usepackage[brazil]{babel}
\usepackage[utf8]{inputenc}

\linespread{1.1}

\newcommand{\sepitem}{\vspace{0.1in}\item}
\newcommand{\titulo}{\item \textbf}
\begin {document}
\lstset{language=C}
\small{
  \title{
    {\small
      Departamento de Ciência da Computação \hfill IME/USP}\\\vspace{0.1in}
    MAC0300 - Métodos Numéricos para Álgebra Linear - 2015/S2
  }
  \vspace{-0.6in}
  \author{
    António Martins Miranda (Nº 7644342) \{\textit{amartmiranda@gmail.com}\} \\
    \and
    António Rui Castro Júnior (Nº 5984327) \{\textit{antonio.castro@usp.br}\}
    \vspace{-0.6in}
  }
  \date{EP2 - Métodos iterativos para sistemas lineares: Gradientes Conjugados}
  \maketitle
}
\vspace {-0.3in}
\thispagestyle{empty}

\setlength{\parindent}{5ex}

\section{Método de Gradientes Conjugados}
\subsection{Descrição}
...
\subsection{Implementação}
\subsubsection{Método de gradientes conjugados}
...
\subsubsection{Gerador de matrizes esparsas definidas positivas}
...
\subsection{Experimentos}
...
\subsection{Análise de resultados}
...

\vfill

\raggedleft
    {\sc Novembro/2015}
    
\end{document}
